\documentclass[a4paper,11pt,titlepage]{article}
\usepackage[utf8]{inputenc}
\usepackage[english]{babel}

\usepackage[margin=1.25in]{geometry}

\usepackage{amsmath,amssymb,amsthm}
\usepackage[sort]{natbib}
\usepackage[colorlinks,allcolors=blue]{hyperref}
\PassOptionsToPackage{usenames,dvipsnames}{xcolor}

\usepackage[small,bf]{titlesec}
\titlelabel{\thetitle.\hspace{0.5em}}
\titleformat{\paragraph}[runin]{\bfseries}{\thetitle}{0em}{}[.]
\titleformat{\subparagraph}[runin]{\it}{\thetitle}{0em}{}[.]
\titlespacing{\paragraph}{0pt}{*3}{*1}
\titlespacing{\subparagraph}{0pt}{*2}{*1}

\usepackage{enumitem}
\setlist[enumerate]{leftmargin=*,itemsep=\lineskip,parsep=0mm,partopsep=0mm,%
  topsep=\lineskip}
\setlist[itemize]{leftmargin=*,itemsep=0.5\parskip,parsep=0mm,partopsep=0mm,%
  topsep=-0.5\parskip,label={{$\bullet$}}}
\setlist[trivlist]{leftmargin=*,itemsep=0.5\parskip,parsep=0mm,partopsep=0mm,%
  topsep=0\parskip}

\renewcommand{\P}{\mathrm{P}}
\newcommand{\F}{\mathcal{F}}
\newcommand{\R}{\mathbb{R}}
\newcommand{\prt}[2]{\frac{\partial{#1}}{\partial{#2}}}
\newcommand{\prtt}[2]{\frac{\partial^2{#1}}{\partial{#2}^2}}
\newcommand{\Prtt}[3]{\frac{\partial^2{#1}}{\partial{#2}\partial{#3}}}
\renewcommand{\epsilon}{\varepsilon}
\renewcommand{\phi}{\varphi}
\renewcommand{\Re}{\mathrm{Re}}
\newcommand{\Vega}{\mathcal{V}}
\newcommand{\eqd}{\stackrel{\mathrm{d}}{=}}
\DeclareMathOperator{\E}{E}
\DeclareMathOperator{\Var}{Var}
\DeclareMathOperator{\I}{I}
\DeclareMathOperator{\sgn}{sgn}
\DeclareMathOperator{\Law}{Law}
\renewcommand{\hat}{\widehat}
\renewcommand{\tilde}{\widetilde}

\theoremstyle{remark}
\newtheorem{remark}{Remark}


\title{\bfseries {\LARGE\texttt{vol} package}\\[1em]
  \Large Mathematical formulas}
\author{}
\date{}


\begin{document}
\maketitle

\tableofcontents\newpage

\section{Models based on a geometric Brownian motion}
\subsection{Black--Scholes model}
The base asset in the model is stock which under the pricing measure has the dynamics
\[
dS_t = rS_t dt + \sigma S_t dW_t, \qquad S_0 = s > 0,
\]
where $r\in \R$ is the risk-free interest rate and $\sigma>0$ is the volatility.
We assume zero dividend rate.

\paragraph{Call option price \citep{BlackScholes73}} 
The price at time $t=0$ of a call option with expiration time $T$ and strike $K$:
\[
C := e^{-rT} \E (S_T-K)^+ = s\Phi(d_1) - e^{-rT}K\Phi(d_2),
\]
where 
\[
d_1 = \frac{1}{\sigma\sqrt T} 
  \left( \ln\frac{s}{K} + \left(r + \frac{\sigma^2}{2} \right) T  \right),
\quad 
d_2 = d_1 - \sigma\sqrt T = \frac{1}{\sigma\sqrt T} 
  \left( \ln\frac{s}{K} + \left(r - \frac{\sigma^2}{2} \right) T \right).
\]

\paragraph{Put--call parity}
The prices of a call option and a put option with the same strike and expiration time satisfy the relation
\[
C - P = s - e^{-rT}K.
\]

\paragraph{Greeks}
Some common Greeks of a call option:
\begin{align*}
&\Delta := \prt Cs = \Phi(d_1),\\
&\Vega := \prt S\sigma = s\phi(d_1)\sqrt{T},\\
&\Theta := -\prt CT 
  = -\frac{s\phi(d_1)\sigma}{2\sqrt T} - rKe^{-rT}\Phi(d_2),\\
&\Gamma := \prtt Cs = \frac{\phi(d_1)}{s\sigma\sqrt T}.
\end{align*}

\paragraph{Implied volatility approximation}
\cite{Brenner88} approximation of implied volatility of a call options with strike $K\approx se^{rT}$:
\[
\hat\sigma \approx \frac{C}{s} \sqrt{\frac{2\pi}{T}}
  \approx \frac{2.5C}{s\sqrt T}.
\]
It is obtained by setting $K = se^{rT}$, $\Phi(x) = \frac12 + \frac{1}{\sqrt{2\pi}} x$ in the Black--Scholes formula.

A more accurate approximation by \cite{CorradoMiller96}:
{\small
\[
\hat\sigma 
  \approx \frac{\sqrt{2\pi}}{\sqrt T (s + e^{-rT}K)} 
    \left( 
      C - \frac{s-e^{-rT}K}{2} 
      + \sqrt{
          \left( \left( C - \frac{s-e^{-rT}K}{2} \right)^2 
          - \frac{(s-e^{-rT}K)^2}{\pi} \right)^+ 
        } 
    \right).
\]
}

\subsection{Black model}
The base asset in the model is a forward or futures contract which under the pricing measure has the dynamics
\[
dF_t = \sigma F_t dW_t, \qquad F_0=f>0,
\]
where $\sigma>0$ is the volatility.

In what follows, $r\in \R$ denotes the risk-free interest rate, $T$ denotes the expiration time of an option, and $T'$ denotes the expiration (delivery) time of a forward contract.
For options on futures contracts, the expiration time of a futures contract does not enter the formulas.

\paragraph{Call option price \citep{Black76}} 
The price of a call option with time to expiration $T$ and strike $K$ on a futures contract:
\[
C_\mathrm{fut} := e^{-rT} \E(F_T - K)^+ = e^{-rT}(f\Phi(d_1) - K\Phi(d_2)).
\]
The price of a call option on a forward contract:
\[
C_\mathrm{for} := e^{-rT'} \E(F_T - K)^+ = e^{-rT'}(f\Phi(d_1) - K\Phi(d_2))
\]
(the difference is in the discounting factor).
In both formulas,
\[
d_1 = \frac{1}{\sigma\sqrt T} 
  \biggl( 
    \ln\frac{f}{K} + \frac{\sigma^2}{2} T
  \biggr),\qquad 
d_2 = d_1 - \sigma\sqrt T 
  = \frac{1}{\sigma\sqrt T} \biggl(\ln\frac{f}{K} 
    -\frac{\sigma^2}{2} T\biggr).
\]

\paragraph{Put--call parity}
The prices of a call option and a put option with the same strike and expiration time satisfy the relations
\[
C_\mathrm{fut} - P_\mathrm{fut} = e^{-rT}(f - K), \qquad
C_\mathrm{for} - P_\mathrm{for} = e^{-rT'}(f - K).
\]

\paragraph{Greeks}
Some common Greeks of a call option:
\begin{align*}
&\Delta_\mathrm{fut}  = e^{-rT}\Phi(d_1),&
&\Delta_\mathrm{for}  = e^{-rT'}\Phi(d_1),\\
&\Vega_\mathrm{fut}   = e^{-rT}f  \phi(d_1)\sqrt T,&
&\Vega_\mathrm{for}   = e^{-rT'}f  \phi(d_1)\sqrt T,\\
&\Theta_\mathrm{fut}  = r C_\mathrm{fut} - 
                          e^{-rT}\frac{f\phi(d_1)\sigma}{2\sqrt T},&
&\Theta_\mathrm{for}  = -e^{-rT'}\frac{f\phi(d_1)\sigma}{2\sqrt T},\\
&\Gamma_\mathrm{fut}  = e^{-rT}\frac{\phi(d_1)}{f\sigma\sqrt T},&
&\Gamma_\mathrm{for}  = e^{-rT'}\frac{\phi(d_1)}{f\sigma\sqrt T},
\end{align*}
where $\Delta = \partial C/\partial f$,
$\mathcal{V} = \partial C/\partial \sigma$,
$\Theta = -\partial C/\partial T$,
$\Gamma = \partial^2 C/ \partial f^2$.
% A convenient relation: f\phi(d_1) = K\phi(d_2).

\paragraph{Implied volatility approximation} 
Similarly to \cite{Brenner88}, for a call option with strike $K\approx f$ we have
\[
\hat\sigma 
  \approx  e^{rT}\frac{C_\mathrm{fut}}{f} \sqrt{\frac{2\pi}T} 
  \approx e^{rT} \frac{2.5C}{f\sqrt{T}}, \qquad
\hat\sigma 
  \approx e^{rT'}\frac{C_\mathrm{for}}{f} \sqrt{\frac{2\pi}T}
  \approx e^{rT'} \frac{2.5C}{f\sqrt{T}}.
\]

\paragraph{A remark on reduction of the Black model to the Black--Scholes model}
When programming these models, it is be desirable to have a single set of functions (or a class)  instead of two separate ones for the Black-Scholes and Black models.
The formulas below show how computation in the Black model can be reduced to the Black--Scholes model.

Let $C^{(0)}$, $\Delta^{(0)}$, etc.\ denote the option price and Greeks in the Black--Scholes model with zero interest rate and initial price $s$ being equal to the forward/futures price $f$; the volatility $\sigma$ being the same as in the Black model.

Consider an option in the Black model with interest rate $r\in \R$ and let $\tau$ be the time to delivery of the base asset, which is $\tau=T$, the option's expiration time, if the base asset is a futures contract\footnote{Because futures contract are marked to market, effectively the base asset is delivered when the option is exercised.}, or $\tau=T'\ge T$ if the base asset is a forward contract.

Then we have the following formulas:
\begin{align*}
&C      =  e^{-r\tau} C^{(0)},\\
&\Delta = e^{-r\tau}\Delta^{(0)},\\
&\Vega  = e^{-r\tau} \Vega^{(0)},\\
&\Theta =
  \begin{cases}
    e^{-r\tau} \Theta^{(0)}      &\text{for a forward contract},\\
    e^{-r\tau} \Theta^{(0)} + rC &\text{for a futures contract},
  \end{cases}\\
&\Gamma     = e^{-r\tau}\Gamma^{(0)},\\
&\hat\sigma = \hat\sigma^{(0)}.
\end{align*}


\section{CEV (Constant Elasticity of Variance) model}
(A concise exposition of the theory below can be found in \cite{LinetskyMendoza10})
The base asset has the following dynamics under the pricing measure:
\[
dS_t = rS_t dt + \sigma S_t^\beta dW_t, \qquad S_0 = s > 0,\\
\]
where $r\in\R_+$ is the risk-free interest rate, and $\sigma>0$, $\beta\ge 0$ are parameters of the model. In what follows, assume that $\beta\neq 1$ (otherwise we have the Black--Scholes model).
For convenience, denote
\[
\nu = \frac{1}{2(\beta-1)}.
\]
If $\beta\in[0,1)$, the process may reach zero (see below).
We will assume that the process remains at zero forever after reaching it, e.g.\ the company bankrupts.

The case $r\neq 0$ can be reduced to $r=0$ by the following transformation:
\begin{equation}
\label{cev-drift-change}
S_t^{(r)} \eqd e^{rt} S_{\tau(t)}^{(0)}, \qquad
  \text{where}\ \tau(t) = \frac{e^{2r(\beta-1)t} -1}{2r(\beta-1)}.
\end{equation}
Another useful fact is that if $r=0$, then the process
\begin{equation}
\label{cev-bessel}
X_t = \frac{S_t^{2(1-\beta)}}{\sigma^2(1-\beta)^2}.
\end{equation}
is a squared Bessel process of dimension $\delta=\frac{1-2\beta}{1-\beta}$, i.e.\ satisfies the SDE
\[
dX_t = \delta dt + 2\sqrt{X_t} dW_t.
\]

\paragraph{Path and distributional properties}
The following properties can be derived from the known transition distribution of the squared Bessel process (see section ``Simulation'' below).

\subparagraph{Case $\beta\in[0,1)$}
\begin{itemize}
\item The process $S_t$ reaches zero in finite time with positive probability, and the distribution of the reaching time $\tau$ is
\[
\P(\tau \le t) 
  = G \left( 
    |\nu|,\ \frac{rs^{2(1-\beta)}}{\sigma^2(1-\beta)(1-e^{2r(\beta-1)t})} 
  \right),
\]
where $G(\nu,x) = \frac{1}{\Gamma(\nu)} \int_x^\infty u^{\nu-1}e^{-u}du$ is the complementary gamma distribution function (use \texttt{scipy.special.gammaincc($\nu$,\,$x$)}).
For $r=0$, the second argument of $G$ should be understood in the limit $r\to0$, i.e.\ we have
\[
\P(\tau\le t) = G \left(
  |\nu|,\ \frac{s^{2(1-\beta)}}{2\sigma^2(1-\beta)^2t} 
\right) 
\qquad \text{(if $r=0$)}.
\]

\item The density of the absolutely continuous part of the transition distribution) is given by (for $x>0$)
\begin{equation}
\label{cev-density}
\P(S_t^{(0)}\in dx \mid S_0=s) = 
  \frac{x^{-2\beta+\frac12}\sqrt s}
       {\sigma^2|\beta-1|t} 
  I_{|\nu|} \left( 
    \frac{(sx)^{1-\beta}}{\sigma^2(\beta-1)^2 t}
  \right) 
  \exp \left( 
    -\frac{s^{2-2\beta} + {2-2\beta}}{2\sigma^2(\beta-1)^2t} 
  \right),
\end{equation}
where $I_\nu(z)$ is the modified Bessel function of the first kind of order $\nu$ (use \texttt{scipy.\allowbreak special.iv($\nu$,\,$z$)}).
Note that the integral of this density over $\R_+$ is less than 1, and the remaining probability is the mass at zero.

\item The process $S_t$ is a martingale.
\end{itemize}

\subparagraph{Case $\beta>1$}
\begin{itemize}
\item The process never reaches zero.

\item The transition density is given by the same formula (\ref{cev-density}), but now it integrates to 1.

\item The process $S_t$ is a strict local martingale and
\begin{align*}
&\E S_t^{(r)} = e^{rt}s \left( 1 - 
  G\left(
    \nu,\ 
    \frac{rs^{2-2\beta}}{\sigma^2(\beta-1)(e^{2r(\beta-1)t}-1)} 
  \right) 
\right) \qquad \text{if}\ r>0,\\
&\E S_t^{(0)} = s \left( 1 - 
  G \left( \nu,\ \frac{s^{2-2\beta}}{2\sigma^2(\beta-1)^2t} \right)
\right).
\end{align*}
\end{itemize}

\paragraph{Call option price}
Let $C(T,K) = e^{-rT}\E(S_T-K)^+$. Then we have
\begin{align*}
\text{for $\beta\in(0,1)$:}&\quad 
  C(T,K) = s Q(y; 2(1+|\nu|), \xi) - e^{-rT}K F(\xi; 2|\nu|, y),\\
\text{for $\beta>1$:}&\quad 
  C(T,K) = s Q(\xi; 2\nu, y) - e^{-rT}K F(y; 2(1+\nu), \xi),
\end{align*}
where $F(z; d,\lambda)$ and $Q(z; d,\lambda) = 1 - F(z;d,\lambda)$ are the distribution and survival functions of the non-central chi-square distribution with $d$ degrees of freedom and non-centrality parameter $\lambda$ (use \texttt{scipy.stats.ncx2($d$,\,$\lambda$)}), and
\[
\xi = \frac{2r s^{2(1-\beta)}}{\sigma^2(1-\beta)(1 - e^{2r(\beta-1)T})},
\qquad
y = \frac{2rK^{2(1-\beta)}}{\sigma^2(1-\beta)(e^{2r(1-\beta)T} - 1)}.
\]
For $r=0$ we obtain $\xi$ and $y$ by passing to the limit $r\to 0$:
\[
\xi = \frac{s^{2(1-\beta)}}{\sigma^2(1-\beta)^2T}, \qquad
y = \frac{K^{2(1-\beta)}}{\sigma^2(1-\beta)^2T} \qquad\text{(if $r=0$)}.
\]
Note that thanks to formula \eqref{cev-drift-change}, we can reduce pricing of options with arbitrary $r>0$ to the case $r=0$:
\begin{equation}
\label{cev-price-drift-change}
C^{(r)}(T,K) = C^{(0)}(T', K'), \qquad 
T' = \tau(T) = \frac{e^{2r(\beta-1)T}-1}{2r(\beta-1)}, \qquad 
K' = e^{-rT} K.
\end{equation}

\paragraph{Approximation of implied volatility}
The Black--Scholes implied volatility produced by the CEV model can be approximated by the formula of \cite{HaganWoodward99}.
If $r=0$, then
\begin{equation}
\label{cev-approximate-iv}
\hat\sigma(T,K) = \frac{\sigma}{\tilde s^{1-\beta}} 
  \biggl(
    1 + \frac{(1-\beta)(2+\beta)}{24} \biggl(\frac{s-K}{\tilde s}\biggr)^2
    + \frac{(1-\beta)^2}{24}\frac{\sigma^2T}{\tilde s^{2(1-\beta)}} + \ldots
  \biggr),
\end{equation}
where $\tilde s = \frac12(s+K)$.

If $r>0$, then with $T' = \tau(T)$, $K' = e^{-rT} K$ (as in \eqref{cev-price-drift-change}), we have
\[
\hat \sigma^{(r)}(T,K) = \hat \sigma^{(0)}(T', K')\sqrt{\frac{T'}T}.
\]

\paragraph{Simulation}
Simulation can be done from the exact transition density.
It is easier to simulate the squared Bessel process $X_t$ from \eqref{cev-bessel}, then get $S^{(0)}$ and finally transform to $S_t^{(r)}$:
\[
S_t^{(r)} = e^{rt} (\sigma^2(1-\beta)^2 X_{\tau(t)})^{\frac1{2(1-\beta)}}.
\]
To simulate $X_t$, we use the following transition density:
\begin{align*}
&\text{for $\beta>1$:}\quad 
  \P(X_{t+\Delta t} \le x\mid X_t=y) 
    = \P\left(
        \chi'^2\left(\delta, \frac y{\Delta t}\right) 
        \le \frac{x}{\Delta t}
      \right)\\
&\text{for $\beta<1$:}\quad 
  \P(X_{t+\Delta t} \le x\mid X_t=y) 
    = \P\left(
        \chi'^2\left(2-\delta, \frac x{\Delta t}\right) 
        \ge \frac{y}{\Delta t}
      \right),
\end{align*}
where $\chi'^2(d,\lambda)$ is the non-central chi-square distribution with $d$ degrees of freedom and non-centrality parameter $\lambda$.
In Python, if $\beta>1$, it is convenient to use \texttt{scipy.stat.ncx2(df=$\delta$,\, nc=$\frac{x}{\Delta t}$,\, scale=$\Delta t$).rvs(\ldots)}.
If $\beta<1$, we have to simulate a uniform random variable and invert the distribution function.


\section{Local volatility}
Assume the base asset is stock. Let $S_0=s$, and $r$ be the (constant)
risk-free interest rate. Denote by $\hat C(T,K)$ the prices of call options
observed in the market, which are assumed to be available for all
$T\in[0,T_{\max}]$ and $K>0$.

By \emph{local volatility} we call a function $\sigma(t,s)$ such that the
option prices $C(T,K) = e^{-rT}\E(S_T-K)^+$ produced by the model
\begin{equation}
\label{localvol-sde}
dS_t = rS_t dt + \sigma(t,S_t) S_t dW_t, \qquad S_0 = s,
\end{equation}
coincide with the market prices, i.e.\ $C(T,K) = \hat C(T,K)$ for all
$T\in[0,T_{\max}]$, $K>0$. In what follows, assume that such $\sigma(t,s)$
exists, and equation \eqref{localvol-sde} has a unique solution such that
$e^{-rt}S_t$ is a martingale.

In a similar way, if the base asset is a futures of forward contract, by local
volatility we call a function $\sigma(t,f)$ such that the model
\[
dF_t = \sigma(t,F_t)F_t dW_t, \qquad F_0 = f,
\]
produces option prices equal to the market prices.

\paragraph{Dupire's formula \citep{Dupire94}} 
The function $\sigma(t,s)$ can be found from the equation
\[
\sigma^2(t,s) = \frac{2 C'_T(t,s) + rs C'_K(t,s)}{s^2 C''_{KK}(t,s)}.
\]
If the base asset is a forward contract, we have the same formula as for
$r=0$, i.e.
\[
\sigma^2_{\mathrm{for}}(t,f) = \frac{2C'_T(t,f)}{f^2 C''_{KK}(t,f)}.
\]
If the base asset is a futures contract, then
\[
\sigma^2_{\mathrm{fut}}(t,f) = 
\frac{2(C'_T(t,f) + r C(t,f))}{f^2 C''_{KK}(t,f)}.
\]

In practical applications, one need to use market prices $\hat C$ in place of
$C$ and find the derivatives of the function $\hat C(T,K)$ numerically
(perhaps, first interpolating the price surface with some smooth or piecewise
smooth function).

\begin{remark} 
It is necessary to impose some technical conditions for the validity of
Dupire's formula. For example it is sufficient to require that the solution
$S_t$ of \eqref{localvol-sde} has a continuous density $f(t,s)$ for each
$t\in(0,T_{\max}]$ and the function $C(T,K)$ is in the class $C^{1,2}$.
\end{remark}


\paragraph{Approximation of implied volatility} Assume the interest rate is
zero and the base asset price has the dynamics
\[
dS_t = A(S_t) dW_t, \qquad S_0 = s > 0.
\]
Then the Black--Scholes implied volatility produced by this model can be
approximated by the following formula (\cite{HaganWoodward99}; formula
\eqref{cev-approximate-iv} for the CEV model is a particular case of it):
\begin{multline*}
\hat\sigma(T,K) = \frac{ A(\tilde s)}{\tilde s} 
\biggl\{ 1 + \frac1{24} 
  \biggl[ 
     \frac{A''(\tilde s)}{A(\tilde s)} -
     2\biggl( \frac{A'(\tilde s)}{A(\tilde s)} \biggr)^2 + \frac{2}{\tilde s^2}
  \biggr] (s-K)^2 \\ + 
  \frac{1}{24} 
  \biggl[ 
     2\frac{A''(\tilde s)}{A(\tilde s)} - 
     \biggl( \frac{A'(\tilde s)}{A(\tilde s)} \biggr)^2 + 
    \frac{1}{\tilde s^2}
  \biggr]  A^2(\tilde s)T + \ldots
\biggl\}
\end{multline*}
where $\tilde s = \frac12(s+K)$. 


\section{Heston model}
The base asset  has the following dynamics under the pricing measure:
\begin{align*}
&dS_t = r S_t dt + \sqrt{V_t} dW_t^1, \qquad S_0 = s > 0,\\
&dV_t = \kappa(\theta-V_t)dt + \sigma\sqrt{V_t} dW_t^2, \qquad V_0 = v > 0,\\
&dW_t^1 dW_t^2 = \rho dt,
\end{align*}
where $r\in\R$ is the risk-free interest rate, and $\kappa,\theta,\sigma > 0$,
$\rho\in(-1,1)$ are parameters.

The variance process $V_t$ is strictly positive for all $t\ge 0$ if Feller's
condition hold:
\[
2\kappa\theta \ge \sigma^2;
\]
otherwise it is non-negative and reflects at zero.


\subsection{Heston's and Lewis' semi-closed formulas for option prices }
\paragraph{Heston's formula \citep{Heston93}} 
The price of a call option with expiration time $T$ and strike $K$:
\[
C = \frac{s-e^{-rT K}}{2} 
+ \frac1\pi \int_0^\pi \Re\left(
  \frac{e^{-u\ln K} s \tilde\phi(u) - e^{-rT}K \phi(u)}{iu}
\right) du, 
\]
where $\phi(u) = \E e^{iu\ln X_T}$ is the characteristic function of the
log-price $X_T = \ln S_T$ at expiration time under the martingale measure, and
$\tilde\phi(u)$ is the same characteristic function but under the measure such
that $e^{rt}/S_t$ is a martingale. These characteristic functions are related
as follows:
\[
\tilde\phi(u) = \frac{\phi(u-i)}{s e^{rT}}.
\]
The function $\phi(u)$ is given by the formula (with $i=\sqrt{-1}$ everywhere)
\[
\phi(u) = \exp(C(u) + D(u)v + iu \ln s),
\]
where
\begin{align*}
&C(u) = irT u + \frac{\kappa\theta}{\sigma^2}
  \left(
    (\kappa- i\rho\sigma u - d(u)) T -
    2\ln\left( \frac{1-g(u)e^{-d(u)T}}{1-g(u)} \right)
  \right),\\
&D(u) = \frac{\kappa - i\rho\sigma u - d(u)}{\sigma^2}
  \left( \frac{1-e^{-d(u)T}}{1-g(u)e^{-d(u)T}} \right),
\end{align*}
and
\[
d(u) = \sqrt{(i\rho\sigma u - \kappa)^2 + \sigma^2(iu + u^2)}, \qquad
g(u) = \frac{i\rho\sigma u- \kappa +d(u)}{i\rho\sigma u - \kappa -d(u)}.
\]
\begin{remark}
The above formulas are obtained by choosing the good (stable) solution of the
Riccati equation arising in the derivation of Heston's formula, see
\cite{Albrecher+07}.
\end{remark}

\paragraph{Lewis' formula \citep{Lewis00}}
The price of a call option with expiration time $T$ and strike $K$:
\[
C = s - \frac{Ke^{-rT}}{2\pi} 
\int_{-\infty+i/2}^{+\infty+i/2} e^{-ixu} \frac{\hat H(u)}{u^2 - iu} du,
\]
where $x=\ln(s/K)+rT$ and $\hat H(u)$ is the \emph{fundamental transform} for
the Heston model, which is given by
\[
\hat H(u) = \exp(f_1(u) + vf_2(u)),
\]
where
\begin{align*}
&f_1(u) = \frac{2\kappa\theta}{\sigma^2}
  \left[ 
    qg(u) - \ln\left( \frac{1-h(u)e^{-q\xi(u)}}{1-h(u)} \right)
  \right],\\
&f_2(u) = \left( \frac{1-e^{-q\xi(u)}}{1-h(u)e^{-q\xi(u)}} \right)g(u),
\end{align*}
and
\begin{align*}
&g(u) = \frac{b(u) - \xi(u)}{2}, \quad 
h(u)  = \frac{b(u) - \xi(u)}{b(u) + \xi(u)}, \quad 
q     = \frac{\sigma^2T}{2},\\
&\xi(u) = \sqrt{b(u)^2 + \frac{4(u^2-iu)}{\sigma^2}}, \qquad 
b(u)    = \frac{2(i\rho\sigma u + \kappa)}{\sigma^2}.
\end{align*}
\begin{remark}
The above formulas are from \citet[Ch.~2]{Baustian+17}. They can be also found
in \cite{Lewis00}, but in a somewhat less compact form.
\end{remark}


\subsection{Simulation methods}
Suppose we need to simulate the values of the processes $(S_t, V_t)$ at points
$t_i = i \Delta t$, $i=0,\ldots,n$, where $\Delta t>0$ is a time step.

\paragraph{Euler's scheme} 
In order to avoid negative values of the variance process, standard Euler's
scheme can be modified as follows: denote $X_t = \ln S_t$ and simulate
\begin{align*}
&X_{t_0}     = \ln s, \quad V_{t_0} = v,\\
&X_{t_{i+1}} = X_{t_i} + \left(r -  \frac{V_{t_i}^+}2\right)\Delta t 
  + \sqrt{ V_{t_i}^+} \left(\rho Z_{i+1} + \sqrt{1-\rho^2} Z'_{i+1}\right)
    \sqrt{\Delta t}, \\
&V_{t_{t+i}} =  V_{t_i} + \kappa(\theta -  V_{t_i}^+) \Delta t +
  \sigma \sqrt{V_{t_i}^+} Z'_{i+1} \sqrt{\Delta t},
\end{align*}
where $Z_i$ and $Z'_i$ are independent sequences of i.i.d.\ standard normal
variables. Then we recover $S_{t_{i}} = \exp(X_{t_i})$,

\paragraph{Exact scheme \citep{BroadieKaya06}} 
The idea of this scheme is based on the representation
\begin{align}
\label{broadie-kaya-1}
&S_{t_{i+1}} = 
  S_{t_i} 
  \exp\left( 
    r\Delta t - \frac12 \int_{t_i}^{t_{i+1}} V_s  ds 
    + \rho\int_{t_i}^{t_{i+1}} \sqrt{V_s} d W_s^1 
    + \sqrt{1-\rho^2} \int_{t_i}^{t_{i+1}} \sqrt{V_s} d W_s^2
  \right),\\
\label{broadie-kaya-2}
&V_{t_{i+1}} = V_{t_i} + \kappa\theta\Delta t 
  - \kappa\int_{t_i}^{t_{i+1}} V_s ds 
  + \sigma \int_{t_i}^{t_{i+1}} \sqrt{V_s} d W_s^1,
\end{align}
where $W^1_t$ and $W_t^2$ are independent Brownian motions. To pass from $t_i$
to $t_{i+1}$, perform the following steps:
\begin{enumerate}
\item simulate $V_{t_{i+1}}$ given the value of $V_{t_i}$,
\item simulate $I_{i+1} := \int_{t_i}^{t_{i+1}} V_s ds$ given the values of
  $V_{t_i}$ and $V_{t_{i+1}}$,
\item simulate $J^1_{i+1} := \int_{t_i}^{t_{i+1}} \sqrt{V_s} d W_s^1$ and 
  $J^2_{i+1} := \int_{t_i}^{t_{i+1}} \sqrt{V_s} d W_s^2$,
\item express $S_{t_{i+1}}$ through the simulated variables.
\end{enumerate}

\subparagraph{Step 1} It is known that
\[
\Law(V_{t_{i+1}} \mid V_{t_i}) = 
  \frac{\sigma^2(1 - e^{-\kappa\Delta t})}{4\kappa}
  \chi_d'^2\biggl(
    \frac{4\kappa e^{-\kappa\Delta t}}
         {\sigma^2(1-e^{-\kappa \Delta t})}V_{t_i}
  \biggr), \qquad 
d = \frac{4\theta\kappa}{\sigma^2},
\]
where $\chi_d'^2(\lambda)$ is the non-central chi-square distribution with $d$
degrees of freedom and non-centrality parameter $\lambda$. Hence, we can sample
from this distribution to get $V_{t_{i+1}}$.

\subparagraph{Step 2 (the most difficult)} Let $F(x)$ denote the conditional
distribution of $I_{i+1}$ (yet to be found):
\[
F(x) = \P(I_{i+1} \le x \mid V_{t_i}, V_{t_{i+1}}).
\]
We can simulate $U_{i+1}$ from a sequence of i.i.d.\ uniform random variables
and numerically solve the equation $F(x) = U_{i+1}$. Then let $I_{i+1} = x^*$
for the solution (i.e.\ use Smirnov's transform).

To compute $F(x)$, we invert the conditional characteristic function, which is
known is a closed form. Namely, let
\[
\gamma(u) = \sqrt{\kappa^2 - 2\sigma^2 i u}, \qquad
c_1       = e^{-\kappa \Delta t}, \qquad 
c_2(u)    = e^{-\gamma(u) \Delta t}.
\]
Then
\begin{multline*}
\phi(u) := \E(e^{i u I_{i+1}} \mid V_{t_i}, V_{t_{i+1}}) 
  = \frac{\gamma(u)\sqrt{c_2(u)/c_1}(1 - c_1)}{\kappa(1 - c_2(u))} \\
    \times \exp\left( 
      \frac{V_{t_{i+1}} + V_{t_i}}{\sigma^2} 
      \left[ 
        \frac{\kappa(1+c_1)}{1-c_1} - \frac{\gamma(u)(1+c_2(u))}{1-c_2(u)}
      \right]
    \right) \\
  \times 
    \frac{I_{0.5d-1}\left( \sqrt{V_{t_i}V_{t_{i+1}}c_2(u)}
            \frac{4\gamma(u)}{\sigma^2(1-c_2(u))} \right)}
         {I_{0.5d-1}\left( \sqrt{V_{t_i}V_{t_{i+1}}c_1} 
           \frac{4\kappa}{\sigma^2(1-c_1)} \right)},
\end{multline*}
where $d = 4\theta\kappa/\sigma^2$ is as above, and $I_\nu(x)$ is the modified
Bessel function of the first kind (use \texttt{scipy.special.iv($\nu$,\,$x$)}).

The inversion procedure is based on the formula
\[
F(x) = \frac2\pi \int_0^\infty \frac{\sin(ux)}{u} \Re(\phi(u)) du.
\]
To compute the integral (note: \texttt{scipy.integrate.quad} works pretty bad
here), use the approximation
\[
F(x) \approx 
\frac{hx}\pi + \frac{2}{\pi} \sum_{j=1}^N \frac{\sin(hjx)}{j} \Re(\phi(hj)).
\]
In order to guarantee precision $\epsilon$, the parameters $h>0$ and
$N\in\mathbb{N}$ should be chosen such that
\begin{align*}
&h \ge \frac{\pi}{u_\epsilon}, \qquad 
  \text{where}\ 1 - F(u_\epsilon) = \epsilon,\\
&\frac{|\phi(hN)|}{N} < \frac{\pi\epsilon}{2}.
\end{align*}
The second condition here poses no problem when the summation in the integral
approximation is performed in a loop -- we stop at first $j$ such that $N=j$
satisfies the second condition (and, additionally, to avoid a huge loop, stop
when $j$ becomes quite large, e.g.\ $j=1000$). The first condition is tough,
Broadie and Kaya suggest to find $u_\epsilon$ simply large enough, for example
$u_\epsilon = m + 5 s$, where $m,s$ are the mean and standard deviation of the
distribution $F$, which can be found by numerical differentiation of the
characteristic function (use \texttt{scipy.misc.derivative}). That is, we put
\[
h = \frac{\pi}{m + 5s}, \qquad 
m = \frac{\phi'(0)}{i}, \qquad 
s = \sqrt{-\phi''(0)}.
\]

\subparagraph{Step 3} Put
\begin{align*}
&J_{i+1}^1 = \frac{1}{\sigma} 
  (V_{t_{i+1}} - V_{t_i} - \kappa\theta\Delta t + \kappa I_{i+1}),\\
&J_{i+1}^2 = \sqrt{I_{i+1}} Z_{i+1},
\end{align*}
where $Z_{i+1}$ is a sequence of i.i.d.\ standard normal variables independent
of $V_t$. Note that the formula for $J_{i+1}^2$ follows from the observation
that $W_t^2$ is independent of $V_t$, hence $J_{i+1}$ has the normal
distribution with variance $\int_{t_i}^{y_{i+1}} V_s ds = I_{i+1}$.

\subparagraph{Step 4} Finally, put
\[
S_{t_{i+1}} = S_{t_i}
\exp\left( 
  r\Delta t - \frac12 I_{i+1} + \rho J^1_{i+1} + \sqrt{1-\rho^2} J^2_{i+1}
\right).
\]

\paragraph{QE scheme \citep{Andersen08}}
(QE means ``quadratic-exponential'') This scheme has the same Steps 3--4 as in
the exact scheme, but modifies Steps 1--2 as follows.

\subparagraph{Step 1} We simulate $V_{t_{i+1}}$ from an approximation of its
conditional distribution given the value $V_{t_i}$. Define
\begin{align*}
&m     = \theta + (V_{t_i}-\theta) e^{-\kappa \Delta t},\\
&s^2   = \frac{V_{t_i} \sigma^2 e^{-\kappa\Delta t}}{\kappa} (1-
         e^{-\kappa\Delta t}) + \frac{\theta\sigma^2}{2\kappa}
         (1 - e^{-\kappa\Delta t})^2\\
&\psi  = \frac{s^2}{m^2},\\
&b^2   = \frac{2}{\psi} - 1 +\sqrt{4-2\psi},\\
&a     = \frac{m}{a+b^2},\\
&p     = \frac{\psi - 1}{\psi+1},\\
&\beta = \frac{1-p}{m},
\end{align*}
where actually $m = \E(V_{t_{i+1}} \mid V_{t_i})$, $s^2 = \Var(V_{t_{i+1}} \mid
V_{t_i})$. Then consider two cases.
\begin{enumerate}
\item If $\psi \le \frac32$, simulate
\[
V_{t_{i+1}} =  a(b+Z_{i+1})^2,
\]
where $Z_{i}$ is a sequence of i.i.d.\ standard normal variables.

\item If $\psi > \frac32$, simulate $V_{t_{i+1}}$ from the exponential
distribution with mass at zero:
\[
\P(V_{t_{i+1}}=0) = p, \qquad 
\P(V_{t_{i+1}} \in dx) = \beta(1-p)e^{-\beta x}\ \text{for}\ x>0.
\]
This case be implemented by defining
\[
V_{t_{i+1}} =
\begin{cases}
  0,                                    &\text{if}\ U_{i+1} \le p,\\
  \frac1\beta \ln\frac{1-p}{1-U_{i+1}}, &\text{if}\ U_{i+1} > p.
\end{cases}
\]
Here $U_{i}$ is a sequence of i.i.d.\ uniform random variables on $[0,1]$.
\end{enumerate}

\begin{remark}
The idea of the above procedure is that if $V_{t_i}$ is ``large'', then its
conditional distribution can be approximated by a squared normal distribution,
while if $V_{t_i}$ is ``small'', it is better to approximate $V_{t_{i+1}}$ by
an exponential distribution with mass at zero. The parameters $a,b,p,\beta$ are
chosen to match the conditional mean and variance of the approximation with the
true values. It turns out that such a match is possible if $\psi\le 2$ for the
squared normal approximation and $\psi\ge 1$ for the exponential-with-mass
approximation. Hence we use the threshold $\psi=\frac32$ in the above method
(but, in principle, any value in $[1,2]$ will work).
\end{remark}

\subparagraph{Step 2} We simply put
\[
I_{i+1} = \frac12 (V_{t_i} + V_{t_{i+1}}).
\]

\subparagraph{Steps 3 and 4} The formulas here are the same as in
Broadie--Kaya's scheme, but note that the final formula for $S_{t_{i+1}}$ can
be explicitly written as follows:
\[
S_{t_{i+1}} = S_{t_i} 
\exp\left( 
  r\Delta t + K_0 + K_1 V_{t_i} + K_2 V_{t_{i+1}} + \sqrt{K_3 (V_{t_i} 
  + V_{t_{i+1}})} Z_{i+1}' 
\right),
\]
where $Z_i'$ is a sequence of i.i.d.\ standard normal variables (independent of
$Z_i$ and $U_i$), and $K_i$ are constants:
\begin{align*}
&K_0 = -\frac{\rho\kappa\theta}{\sigma}\Delta t,\\
&K_1 = \frac12 \left( \frac{\kappa\rho}{\sigma}-\frac12 \right) \Delta t 
     - \frac{\rho}{\sigma},\\
&K_2 = \frac12 \left(\frac{\kappa\rho}{\sigma}-\frac12\right)\Delta t 
     + \frac{\rho}{\sigma},\\
&K_3 = \frac12 (1-\rho^2)\Delta t.
\end{align*}

\paragraph{E+M scheme \citep{MrazekPospisil17}} 
(E+M means ``exact + Milstein''.) This scheme uses the same Steps~2--4 as the
QE scheme, but in Step~1, to simulate the process $V_t$, it uses the Milstein
scheme as follows:
\[
V_{t_{i+1}} = V_{t_i} + \kappa(\theta- V_{t_i}^+) \Delta t 
+ \sigma \sqrt{V_{t_i}^+} Z_{i+1} \sqrt{\Delta t} 
+ \frac14 \sigma^2(Z_{i+1}^2 -1) \Delta t. 
\]


\section{SABR (Stochastic Alpha, Beta, Rho) model}
Assume the dynamics of the base asset $F_t$ (e.g., a futures of a forward
contract) is defined by the equations
\begin{align*}
&dF_t          = \alpha_t F_t^\beta dW_t^1, \qquad F_0 = f > 0,\\
&d\alpha_t     = \nu\alpha_t dW_t^2,        \qquad \alpha_0 = \alpha > 0,\\
&dW_t^1 dW_t^2 = \rho dt,
\end{align*}
where $\alpha>0$, $\beta>0$, $\rho\in(-1,1)$, $\nu > 0$ are the model
parameters. If $\beta <1$, the process $F_t$ may reach zero; in that case we
will assume that it gets trapped at zero.

\paragraph{Approximation of implied volatility \citep{Hagan+02}} 
The implied volatility produced by the model can be approximated by the formula
\begin{multline*}
\hat\sigma(T,K) = 
  \frac{\alpha}
    {(fK)^{(1-\beta)/2} 
      \left\{ 1+\frac{(1-\beta)^2}{24} \ln^2 \frac fK 
        + \frac{(1-\beta)^4}{1920} \ln^4 \frac fK + \ldots
      \right\}} 
   \cdot \left( \frac{z}{x(z)}\right) \cdot\\
   \cdot \left\{ 
     1 + \left[
       \frac{(1-\beta)^2}{24} \frac{\alpha^2}{(fK)^{1-\beta}} 
       + \frac{\rho\beta\nu\alpha}{4(fK)^{(1-\beta)/2}} 
       + \frac{2-3\rho^2}{24} \nu^2 
     \right] T 
     + \ldots 
   \right\},
\end{multline*}
where
\[
z = \frac\nu\alpha (fK)^{(1-\beta)/2} \ln \frac fK, \qquad 
x(z) = \ln\left\{ \frac{\sqrt{1-2\rho z + z^2} + z - \rho}{1-\rho} \right\}.
\]
For at-the-money options ($K=f$) we have
\[
\hat\sigma(T,f) = \frac{\alpha}{f^{1-\beta}} 
\left\{1 + 
  \left[ \frac{(1-\beta)^2}{24} \frac{\alpha^2}{f^{2-2\beta}} 
    + \frac{\rho\beta\alpha\nu}{4f^{1-\beta}} 
    + \frac{(2-3\rho^2)\nu^2}{24}
  \right] T 
  + \ldots
\right\},
\]
i.e.\ in the above general formula we let $z/x(z)=1$, its limit value as $z\to
0$ (we have $z=0$ when $K=f$).

For $\beta=0$, there is a more accurate formula (with an expansion of the 4th
order, rather than the 2nd order in the general formula):
\[
\hat\sigma(T,f) = \alpha\frac{\ln(f/K)}{f-K}
\cdot \left( \frac{z}{x(z)} \right)
\cdot \left\{ 1 + 
  \left[ \frac{\alpha^2}{24fK} + \frac{2-3\rho^2}{24}\nu^2 \right] T 
  + \ldots
\right\},
\]
where
\[
z = \frac\nu\alpha\sqrt{fK} \ln \frac{f}{K}, \qquad 
x(z) = \ln\left\{ \frac{\sqrt{1-2\rho z + z^2} + z - \rho}{1 - \rho} \right\}.
\]
For $\beta=1$, a more accurate formula (also, with a 4th order expansion) is
\[
\hat\sigma(T,f) = \alpha \cdot \left( \frac{z}{x(z)} \right)
\cdot \left\{ 1 + 
  \left[\frac{\rho\alpha\nu}{4} + \frac{(2-3\rho^2)\nu^2}{24} \right] T
  + \ldots
\right\},
\]
where
\[
z = \frac\nu\alpha \ln \frac{f}{K}, \qquad 
x(z) = \ln\left\{ \frac{\sqrt{1-2\rho z + z^2} + z - \rho}{1 - \rho}\right\}.
\]


\section{SVI (Stochastic Volatility Inspired) model}
The SVI model \citep{Gatheral04} approximates the volatility curve without
making explicit assumptions about dynamics of the asset price process. Fix an
expiration time $T>0$ and let
\[
x = \ln\frac{K}{F_T}
\]
denote the log-moneyness of an option with expiration time $T$ and strike $K(x)
= F_Te^x$, where $F_t = \E(S_t) = e^{rt}S_0$ is the forward price. By $w(x)$
denote the total implied variance of the option, i.e.\
\[
w(x) = \hat \sigma^2(T, K(x)) T,
\]
where $\hat \sigma(T,K)$ is, as usual, the implied volatility.

\paragraph{Three parametrizations of the SVI model} 
The following three parametrizations can be found in the literature (see
\cite{GatheralJacquier14}).

\subparagraph{Raw parametrization} The SVI model in the \emph{raw
parametrization} approximates the total implied variance (for fixed $T$) by the
function
\[
w(x) = a + b\left( \rho(x-m) + \sqrt{(x-m)^2 + \sigma^2} \right),
\]
where $a\in\R$, $b\ge 0$, $|\rho|<1$, $m\in\R$, $\sigma>0$ are parameters of
the model. Note that, of course, the same formula can used to approximate the
implied volatility curve in $(\ln K, \hat\sigma^2)$ coordinates, after a change
of parameters.


\subparagraph{Natural parametrization} This parametrization assumes
\[
w(x) = \Delta + \frac{\omega}{2}
\left( 1 + \zeta\rho(x-\mu) + \sqrt{(\zeta(x-\mu)+\rho)^2 + 1-\rho^2} \right).
\]
where $\omega\ge 0$, $\Delta,\mu\in R$, $|\rho|<1$, $\zeta>0$ are parameters.

The following formulas can be used for conversion between the raw and natural
parametrizations (note that the parameter $\rho$ is the same):
\begin{align*}
&\omega = \frac{2b\sigma}{\sqrt{1-\rho^2}},&
  &a = \delta + \frac12 \omega(1-\rho^2),\\
&\delta = a-\frac12 \omega(1-\rho^2),&
  &b = \frac{\omega\zeta}{2},\\
&\mu = m + \frac{\rho\sigma}{\sqrt{1-\rho^2}},&
  &m = \mu-\frac{\rho}{\zeta},\\
&\zeta = \frac{\sqrt{1-\rho^2}}{\sigma},&
  &\sigma = \frac{\sqrt{1-\rho^2}}{\zeta}.
\end{align*}

\subparagraph{Jump-wing parametrization} The parameters are $v(T)$, $\psi(T)$,
$p(T)$, $c(T)$, $\tilde v(T)$ (this parametrization specifies dependence on
$T$), which are obtained from the raw parameters as follows:
\begin{align*}
&v = \frac{a + b(-\rho m + \sqrt{m^2+\sigma^2})}{T},\\
&\psi = \frac{b}{2\sqrt{Tv}} 
  \left( -\frac{m}{\sqrt{m^2 + \sigma^2}} + \rho\right),\\
&p = \frac{1}{\sqrt{Tv}} b(1-\rho),\\
&c = \frac{1}{\sqrt{Tv}} b(1+\rho),\\
&\tilde v = \frac1T \left( a + b\sigma\sqrt{1-\rho^2} \right).
\end{align*}

To construct a volatility surface from this parametrization, we can keep the
jump-wing parameters fixed, and for each $T$ find the corresponding raw
parameters to get a volatility curve for this $T$. The raw parameters are
obtained as follows. First, define
\[
b     = \frac12 \sqrt{Tv}(c+p), \qquad
\rho  = 1 - \frac{2p}{c+p}, \qquad
\beta = \rho - \frac{2\psi\sqrt{Tv}}{b}.
\]
Then, if $\beta=0$, the remaining raw parameters are
\begin{align*}
&m      = 0,\\
&\sigma = \frac{(v-\tilde v)T}{b(1-\sqrt{1-\rho^2})},\\
&a      = T\tilde v - b\sigma\sqrt{1-\rho^2}.
\end{align*}
If $\beta\neq 0$ and $|\beta|\le 1$, then
\begin{align*}
&m = \frac{(v-\tilde v)T}
          {b(-\rho+\sgn(\alpha)\sqrt{1+\alpha^2} - \alpha\sqrt{1-\rho^2})},
  \qquad\text{where}\ \alpha = \sgn(\beta)\sqrt{\frac1{\beta^2} - 1},\\
&\sigma = \alpha m,\\
&a = T\tilde v - b\sigma\sqrt{1-\rho^2}.
\end{align*}
If $|\beta|>1$, we get a non-convex volatility curve; this case should be
excluded.


\paragraph{Parameter calibration} 
The following two-step procedure was proposed by \cite{Zeliade12} (see also
\cite{Aurell14}). Let us work with the raw parametrization. Introduce the
variable
\[
y(x) = \frac{x-m}{\sigma},
\]
so that the raw parametrization can be rewritten in the form
\[
w(x) = a + dy(x) + cz(x), \qquad z(x) =\sqrt{y(x)^2+1},\\
\]
where
\[
d=\rho b\sigma,\qquad c=b\sigma.
\]
For each $m,\sigma$ we can solve the \emph{inner} optimization problem
\[
\sum_{i=1}^n (a + d y(x_i) + c z(x_i) - w_m(x_i))^2 
\xrightarrow[(a,d,c)\in\mathcal{D}]{} \min,
\]
where $w_m(x_i)$ are observed market total implied variances, and $\mathcal{D}$
is the domain defined by the constraints
\begin{align*}
&0      \le c \le 4\sigma,\\
&|d|    \le c,\\
&|d|    \le 4\sigma-c,\\
&0\le a \le \max_i w(x_i).
\end{align*}
This constraints are derived from the model well-posedness conditions and
conditions for absence of arbitrage (see \cite{Aurell14} for details). The
inner optimization problem is a convex minimization problem on a compact
domain, hence it has a unique solution (Python's
\texttt{scipy.optimize.minimize(method="SLSQP")} works quite well here).

Then we solve the \emph{outer} minimization problem and find optimal
$m,\sigma$:
\[
\sum_{i=1}^n 
  ( w(x_i \mid m,\sigma, a^*(m,\sigma), b^*(m,\sigma), \rho^*(m,\sigma)) 
  - w_m(x_i) )^2
\xrightarrow[(m,\sigma)\in\mathcal{E}]{} \min,
\]
where $a^*,b^*,\sigma^*$ is the solution of the inner minimization problem for
given $m,\sigma$, and $\mathcal{E}$ is the domain
\begin{align*}
&\min_i x_i    \le m      \le \max_i x_i,\\
&\sigma_{\min} \le \sigma \le \max_\sigma.
\end{align*}
The boundaries $\sigma_{\min}$, $\sigma_{\max}$ are specified by the user, a
reasonable choice is $\sigma_{\min} = 10^{-4}$, $\sigma_{\max} = 10$. The
objective function in the outer problem is not nice, hence some global
optimization algorithm should be used.


\bibliographystyle{apalike}
\bibliography{vol}

\end{document}
